\documentclass[12pt,a4paper]{article}
\usepackage[utf8]{inputenc}
\usepackage{amsmath}
\usepackage{amsfonts}
\usepackage{amssymb}
\parskip 2ex
\author{Mokili Isaac Janda RegNo: 16/U/7096/PS StudNo: 216013042}
\title{Waze}
\begin{document}
\maketitle
\begin{abstract}
A brief literature review on Waze - a Google car pooling application
\end{abstract}
\section{About Waze}
Waze is a car pooling application that connects drivers to one another\cite{wang2014driver}, it helps people create local driving communities that work together to improve the quality of everyone's daily driving.\cite{wazeMobile}. That might mean helping them avoid the frustration of sitting in traffic\cite{sinai2014exploiting,silva2013traffic,mcclendon2013google}, cluing them in to a police trap or saving five minutes off of their regular commute by showing them new routes they never even knew about.\cite{mcclendon2013google} 

\subsubsection{Features}
Among the notable features of Waze include but not limited to the following:

Waze allows users see drivers and riders going their way, and choose the people they want to carpool with based on detailed profiles, star ratings, and connections like same workplace and mutual friends—along with price and distance off route. Best matches (closest to one's route) will appear at the top.\cite{wazeBlog}
 
Customization of rides: Waze has added filters so one can choose to ride with coworkers only, or opt to select a driver of the same gender.\cite{wazeBlog}
  
Waze also builds it's own Siri for hands-free voice control as noted by \cite{hardawar2012driving}

\bibliographystyle{ieeetr}
\bibliography{waze}
\end{document}
